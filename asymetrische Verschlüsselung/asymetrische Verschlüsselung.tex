\documentclass[a4paper, 12pt, one column]{article}

%% Language and font encodings. This says how to do hyphenation on end of lines.
\usepackage[ngerman]{babel}
\usepackage{natbib}
\bibliographystyle{harvard}
\bibliography{refs}
\usepackage[utf8x]{inputenc}
\usepackage[T1]{fontenc}


%% Sets page size and margins. You can edit this to your liking
\usepackage[top=1.3cm, bottom=2.0cm, outer=2.5cm, inner=2.5cm, heightrounded,
marginparwidth=1.5cm, marginparsep=0.4cm, margin=2.5cm]{geometry}

%% Useful packages
\usepackage{graphicx} %allows you to use jpg or png images. PDF is still recommended
\usepackage[colorlinks=False]{hyperref} % add links inside PDF files
\usepackage{amsmath}  % Math fonts
\usepackage{amsfonts} %
\usepackage{amssymb}  %



\title{Asymetrische Verschlüsselung}
\author{Lorenz Faber}

\begin{document}
	\maketitle
	
	%\begin{abstract}
	%	Add abstract here
	%\end{abstract}
	
	\section{Einführung}
	
	mit dem Zitat 
	“Crypto will not be broken, it will be bypassed”
	– Adi Shamir
	
	Verschlüsselung wird nicht gebrochen, sie wird umgangen.
	Die asymetrische Verschlüsselung
	
	You can cite using to make a reference in the text. Or for a reference in parenthesis. \cite{latexcompanion,knuthwebsite}
	
	\section{Something}
	
	Aldo ist cool
	
	\subsection{Sub-something}
	
	content here
	
	
	%\bibliography{refs}
	\medskip
	
	\begin{thebibliography}{9}
		\bibitem{latexcompanion} 
		Michel Goossens, Frank Mittelbach, and Alexander Samarin. 
		\textit{The \LaTeX\ Companion}. 
		Addison-Wesley, Reading, Massachusetts, 1993.
		
		\bibitem{einstein} 
		Albert Einstein. 
		\textit{Zur Elektrodynamik bewegter K{\"o}rper}. (German) 
		[\textit{On the electrodynamics of moving bodies}]. 
		Annalen der Physik, 322(10):891–921, 1905.
		
		\bibitem{knuthwebsite} 
		Knuth: Computers and Typesetting,
		\\\texttt{http://www-cs-faculty.stanford.edu/\~{}uno/abcde.html (18.01.2021)}
	\end{thebibliography}
\end{document}